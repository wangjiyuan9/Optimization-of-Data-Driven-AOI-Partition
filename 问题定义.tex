%导言区
\documentclass{article} %book report letter
\usepackage{ctex} %显示中文
\usepackage{amsmath}

\title{问题定义}

%正文区
\begin{document} %环境名称
	\maketitle
	\subsection{问题简介}
	物流区域划分问题是对于物流配送问题,为方便管理、提升效率,会将城市划分成多个片区归不同快递员负责。在物流配送过程中会记录下快递员的行驶轨迹数据、订单的位置派单时间信息,通过分析研究可以借此来优化物流区域的划分。如何根据这些数据合理优化划分区域,使物流配送效率更高,同时保证效益更好,是一个新兴的问题。
	%不需要写意义,只需要写问题输入输出
	
	%正文区首先要加“\begin{环境名称}”在最后加\end{环境名称}。为了将标题显示出来,还需要写“\maketitle”。一直无法解决,但是还是可以看到结果,先将其注释掉,出来pdf的结果窗口后,再将注释去掉编译一遍即可。
	
	\subsection{问题抽象}
	物流区域划分要求划分出的区域没有重叠,同时能够覆盖整个平面,则称为一个有效划分。
	物流区域是一个多边形(不考虑圆形或弧形的边界),可以通过多边形的顶点来表示整个区域。对于总体区域的划分,会构成一个平面图,图中的区域即为划分的区域。
	%分点
	对于一个有效划分$\sigma$,会对应一个平面图$G(V,E)$,其中每个顶点对应划分的区域集合的交点$V=\{v_i|(v_i,x_{v_i},y_{v_i})\}$,每条边对应区域集合的交边 ,图中每个面对应区域集合的某个区域 。
	
	\iffalse
	快递员轨迹信息是一组坐标和时间的集合,对于确定的AOI划分,我们也能够知道每时刻快递员所处AOI号。订单信息则是订单送达的位置与派单时刻以及所处AOI号构成。即为:
	L={(x,y,t,i)|(x,y)\in AOI_i}
	\fi
	
	\begin{itemize}
	区域划分优化即已知原先的区域划分结果F,根据揽件数据L等来进行优化得到新的区域划分F',使得新的区域划分方案更优。所谓更优,即提升物流配送效率,缩短配送所需总时间。设对于一个给定的F,以及快递配送需求$w_1,w_2,...,w_s$,有任务分配及轨迹规划算法T,则$T(F',W)$给出了仿真后的配送总时间,若时间比$T(F,W)$更小,则称为更优。物流配送问题不仅涉及效率,还有其他方面,比如管理复杂度。易知,所划分的区域越多,管理越困难,因此,我们希望划分的AOI数量不要太多,即$N_{AOI}(F)$不能太大。为使区域划分方案更加公平,我们希望每个区域内订单数量尽可能差距不大,对于每个区域$\varGamma$,函数$N(\varGamma)$计算区域内的快递数,则$\frac{1}{N_{AOI}(F)} \underset{\varGamma \in F}{\sum{(N(\varGamma)-\bar{N})^2}}$能够尽可能小。设$N_{travel}(L)$是快递员配送过程中经过的AOI数量,则我们希望快递员不需要在不同AOI之间来回往返,即轨迹中经过的AOI数量不会太多。	
	\end{itemize}
	
	综上,该问题即为:
	\begin{equation}
	\underset{F}{\arg \min} Loss(F,T,W)
	\end{equation}
	\begin{equation}
	\begin{split}
	Loss(F,T,W)= & k_1*T(F,W)+k_2*N_{AOI}(F)+k_3*N_{travel}(L)  \\
	& +k_4*\frac{1}{N_{AOI}(F)}\sum_{\varGamma in F}{(N(\varGamma)-\bar{N})^2}
	\end{split}
	\end{equation}
	
\end{document}